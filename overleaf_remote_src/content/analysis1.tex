\section{Analysis I - draft}

% Grundlagen und Logik___________________________________________________________________________________________________
%%%%%%
\subsection{1. Grundlagen und Logik}

% Zahlen und Vektoren___________________________________________________________________________________________________
%%%%%%
\subsection{2. Zahlen und Vektoren}


% Folgen und Reihen ___________________________________________________________________________________________________
%%%%%%
\subsection{3. Folgen und Reihen}  
Eine \emph{ Zahlenfolge} ist eine Funktion  
\[
a : \mathbb{N}_0 \to \mathbb{C}, \quad n \mapsto a_n := a(n).
\]  

\textbf{Konvergenz einer Folge:}  
Eine Folge \( (x_n) \subset \mathbb{R} \) \textbf{konvergiert} gegen \( A \in \mathbb{R} \), falls  
\[
\forall \varepsilon > 0\; \exists N \in \mathbb{N}:\; \forall n \geq N:\; |x_n - A| < \varepsilon.
\]  
Dann schreiben wir \( \lim_{n \to \infty} x_n = A \) und nennen \( A \) den \textbf{Grenzwert}.

\[
Konvergenz \to Beschränktheit
\]  

\textbf{Beispiel Konvergent:} \( x_n = \frac{1}{n} \Rightarrow \lim\limits_{n \to \infty} x_n = 0 \)

\textbf{Beispiel Divergent:} \( x_n = n \Rightarrow \lim\limits_{n \to \infty} x_n = \infty \)

\textbf{Rechenregeln für konvergente Folgen:}  
Seien \( \lim\limits_{n \to \infty} x_n = A \), \( \lim\limits_{n \to \infty} y_n = B \), \( \alpha \in \mathbb{R} \):

\begin{itemize}
  \item \( \lim\limits_{n \to \infty} (x_n + y_n) = A + B \)
  \item \( \lim\limits_{n \to \infty} (x_n y_n) = AB \)
  \item \( \lim\limits_{n \to \infty} (\alpha x_n) = \alpha A \)
  \item Falls \( x_n \neq 0\ \forall n \), \( A \neq 0 \):  
        \( \lim\limits_{n \to \infty} \left( \frac{1}{x_n} \right) = \frac{1}{A} \)
  \item Falls \( x_n \leq y_n\ \forall n \): \( A \leq B \)
\end{itemize}

\textbf{Monotone Konvergenz:}  
Eine monotone Folge \( (x_n) \subset \mathbb{R} \) konvergiert genau dann, wenn sie beschränkt ist.

\begin{itemize}
  \item Monoton wachsend:  
  \( \lim\limits_{n \to \infty} x_n = \sup \{ x_n \mid n \in \mathbb{N} \} \)

  \item Monoton fallend:  
  \( \lim\limits_{n \to \infty} x_n = \inf \{ x_n \mid n \in \mathbb{N} \} \)
\end{itemize}

\textbf{Euler'sche Zahl:}  
\[
e := \lim\limits_{n \to \infty} \left( 1 + \frac{1}{n} \right)^n
\]

\textbf{Eigenschaften:}
\begin{itemize}
  \item \( e \) ist wohldefiniert (Folge konvergiert).
  \item \( e \approx 2{,}718\ldots \) (irrational, nicht periodisch).
\end{itemize}

\textbf{Obere/Untere Grenzwerte:}  
Für eine beschränkte Folge \( (x_n) \subset \mathbb{R} \) gilt:

\[
\limsup\limits_{n \to \infty} x_n = \lim\limits_{n \to \infty} \left( \sup_{k \geq n} x_k \right)
\]
\[
\liminf\limits_{n \to \infty} x_n = \lim\limits_{n \to \infty} \left( \inf_{k \geq n} x_k \right)
\]

\textbf{Immer:}  
\[
\liminf x_n \leq \limsup x_n
\]
\textbf{Bernoulli-de l'Hôspital:} (Abl. existiert, f und g diffbar um a und $g'(x) \not = 0$)
\[
\lim\limits_{x \to a} \frac{f(x)}{g(x)} = \lim\limits_{x \to a} \frac{f'(x)}{g'(x)}
\]
\textbf{\( e^{\log} \)–Trick:}  
\[
\lim f(x)^{g(x)} = \exp\left( \lim g(x) \cdot \log f(x) \right)
\quad \text{falls } f(x) > 0
\]

\textbf{Cauchy-Folge:}  
\[
(x_n) \subset \mathbb{R} \text{ ist Cauchy } \Leftrightarrow  
\forall \varepsilon > 0\; \exists N \in \mathbb{N}:\; |x_n - x_m| < \varepsilon \; \forall m,n \geq N
\]
\textbf{Satz:}  
\[
(x_n) \text{ konvergiert } \Leftrightarrow \text{Cauchy-Folge}
\]

\textbf{Unendliche Grenzwerte:}  
\[
\lim_{n \to \infty} x_n = \infty \Leftrightarrow  
\forall M > 0\; \exists N \in \mathbb{N}:\; x_n > M\; \forall n \geq N
\]
\[
\lim_{n \to \infty} x_n = -\infty \Leftrightarrow  
\forall M > 0\; \exists N \in \mathbb{N}:\; x_n < -M\; \forall n \geq N
\]

\textbf{Reihe }
\[
\sum_{k=0}^\infty a_k \text{ konvergiert} \Leftrightarrow 
\lim_{n \to \infty} s_n = A \in \mathbb{R},\quad s_n := \sum_{k=0}^n a_k
\]
\[
\textbf{Wenn } \sum a_k \text{ konvergiert, dann } \lim a_k = 0
\]

\textbf{Restlemma:}  
\[
\sum_{k=0}^\infty a_k \text{ konvergiert } \Leftrightarrow  
\sum_{k=N}^\infty a_k \text{ konvergiert } \quad \forall N \in \mathbb{N}
\]
\begin{tabular}{|c|c|c|}
\hline
\textbf{Reihe} & \textbf{Bedingung} & \textbf{Konvergenz} \\
\hline
\( \sum z^k \) & \( |z| < 1 \) & konvergent \\
              & \( |z| \geq 1 \) & divergent \\
\hline
\( \sum \frac{1}{k} \) & — & divergent (harmonisch) \\
\hline
\( \sum \frac{1}{k^\alpha} \) & \( \alpha > 1 \) & konvergent \\
                             & \( \alpha \leq 1 \) & divergent \\
\hline
\( \sum a_k \), \( \sum b_k \) & \( 0 \leq a_k \leq b_k \), \( \sum b_k \) konv. & \( \Rightarrow \sum a_k \) konv. \\
\hline
\( \sum a_k \), \( \sum b_k \) & \( a_k \sim b_k \) und \( \sum b_k \) konv. & \( \Rightarrow \sum a_k \) konv. \\
\hline
\end{tabular}


\textbf{Majorantenkriterium:}  
\[
0 \leq a_k \leq b_k,\; \sum b_k \text{ konv.} \;\Rightarrow\; \sum a_k \text{ konv.}
\]

\textbf{Minorantenkriterium:}  
\[
0 \leq a_k \leq b_k,\; \sum a_k \text{ divergiert} \;\Rightarrow\; \sum b_k \text{ divergiert}
\]

\textbf{Cauchy-Kondensation:}  
\[
(a_k) \downarrow,\; a_k \geq 0:  
\quad \sum a_k \text{ konv.} \;\Leftrightarrow\; \sum 2^k a_{2^k} \text{ konv.}
\]

\textbf{Leibniz-Kriterium (alternierende Reihen)}  
Anwendbar für Reihen der Form:
\[
\sum_{k=0}^\infty (-1)^k a_k \quad \text{oder} \quad \sum_{k=0}^\infty (-1)^{k+1} a_k
\]

\textbf{Leibniz-Kriterium:}  
\(
\sum (-1)^k a_k \text{ konvergiert, falls:}
\)

\begin{tabular}{ll}
\checkmark & \( a_k \geq 0 \) \\
\checkmark & \( a_k \downarrow \) (monoton fallend) \\
\checkmark & \( \lim a_k = 0 \)
\end{tabular}

\vspace{0.5cm}

\textbf{Bsp: } \( \sum_{k=1}^\infty \frac{(-1)^k}{k} \)

\begin{tabular}{ll}
\checkmark & \( \frac{1}{k} \geq 0 \) \\
\checkmark & \( \frac{1}{k+1} < \frac{1}{k} \) \\
\checkmark & \( \lim \frac{1}{k} = 0 \)
\Rightarrow \text{ konvergiert nach Leibniz}
\end{tabular}

\textbf{Wurzelkriterium:}  
\( \alpha := \limsup \sqrt[n]{|a_n|} \)

\[
\alpha < 1 \Rightarrow \sum a_n \text{ konv. absolut}, \quad
\alpha > 1 \Rightarrow \sum a_n \text{ divergiert}
\]

\textbf{Bsp: } \( a_n = \frac{1}{3^n} \Rightarrow \sqrt[n]{|a_n|} = \frac{1}{3} \Rightarrow \alpha = \frac{1}{3} < 1 \)

\( \Rightarrow \sum a_n \) konvergiert absolut

\textbf{Quotientenkriterium:}  
\( \alpha := \lim \left| \frac{a_{n+1}}{a_n} \right| \)

\[
\alpha < 1 \Rightarrow \sum a_n \text{ konv. absolut}, \quad
\alpha > 1 \Rightarrow \sum a_n \text{ divergiert}
\]

\textbf{Bsp: } \( a_k = \frac{(k!)^2}{(2k)!} \)

\[
\frac{a_{k+1}}{a_k} = \frac{((k+1)!)^2}{(2k+2)!} \cdot \frac{(2k)!}{(k!)^2}
= \frac{(k+1)^2 \cdot (k!)^2}{(2k+2)(2k+1)(2k)!} \cdot \frac{(2k)!}{(k!)^2}
= \frac{(k+1)^2}{(2k+2)(2k+1)}
\]

\[
\Rightarrow \lim \frac{(k+1)^2}{(2k+2)(2k+1)} = \frac{1}{4} < 1
\Rightarrow \sum a_k \text{ konvergiert absolut}
\]

\textbf{Zeta-Reihe:}  
\[
\zeta(s) := \sum_{n=1}^\infty \frac{1}{n^s}, \quad s > 1
\]

\textbf{Produktsatz:}  
Wenn \( \sum a_n \), \( \sum b_n \) absolut konvergieren, dann konvergiert auch  
\[
\sum_{n=0}^\infty a_{\alpha_1(n)} b_{\alpha_2(n)}
= \left( \sum a_n \right)\left( \sum b_n \right)
\]
für jede Bijektion \( \alpha: \mathbb{N} \to \mathbb{N} \times \mathbb{N} \)

\textbf{Cauchy-Produkt:}  
Wenn \( \sum a_n \), \( \sum b_n \) absolut konvergieren, dann:
\[
\left( \sum a_n \right) \left( \sum b_n \right)
= \sum_{n=0}^\infty \left( \sum_{k=0}^n a_k b_{n-k} \right)
\]

\textbf{Potenzreihe:}  
\[
\sum_{n=0}^\infty a_n x^n
\]

\textbf{Konvergenzradius:}  
Gegeben Potenzreihe \( \sum a_n x^n \), setze
\[
\rho = \limsup_{n \to \infty} \sqrt[n]{|a_n|} \quad \text{oder} \quad
\rho = \limsup_{n \to \infty} \left| \frac{a_{n+1}}{a_n} \right|, \quad
R = \frac{1}{\rho}
\]

\[
R =
\begin{cases}
0, & \text{falls } \rho = \infty \\
\infty, & \text{falls } \rho = 0 \\
\rho^{-1}, & \text{sonst}
\end{cases}
\]

\textbf{Verhalten:}
\begin{itemize}
  \item \( |x| < R \) \quad absolut konvergent
  \item \( |x| > R \) \quad divergent
  \item \( |x| = R \) \quad einzeln prüfen
\end{itemize}

\vspace{0.5em}

\textbf{Summe und Produkt von Potenzreihen:}  
Falls \( \sum a_n x^n \), \( \sum b_n x^n \) Konvergenzradius \( \geq R \) haben, dann gilt:
\[
\sum (a_n + b_n)x^n, \quad
\sum_{n=0}^\infty \left( \sum_{k=0}^n a_k b_{n-k} \right) x^n
\quad \text{haben ebenfalls Radius } \geq R
\]

\vspace{0.5em}

\textbf{Exponentialfunktion:}
\[
e^x = \sum_{k=0}^\infty \frac{x^k}{k!} = \lim_{n \to \infty} \left(1 + \frac{x}{n}\right)^n
\quad \text{mit Radius } R = \infty
\]

\textbf{Eulerformel:}  
\[
e^{ix} = \cos(x) + i \sin(x)
\quad\Rightarrow\quad
\sin(x) = \frac{e^{ix} - e^{-ix}}{2i},\quad
\cos(x) = \frac{e^{ix} + e^{-ix}}{2}
\]

\textbf{Additionstheoreme:}  
\[
\sin(x + y) = \sin x \cos y + \cos x \sin y
\]
\[
\cos(x + y) = \cos x \cos y - \sin x \sin y
\]

\textbf{Periodizität:}
\[
\begin{aligned}
\sin(x + \tfrac{\pi}{2}) &= \cos(x) & \qquad \cos(x + \tfrac{\pi}{2}) &= -\sin(x) \\
\sin(x + \pi) &= -\sin(x) & \cos(x + \pi) &= -\cos(x) \\
\sin(x + 2\pi) &= \sin(x) & \cos(x + 2\pi) &= \cos(x)
\end{aligned}
\]

\textbf{Wichtige Potenzreihen:}
\begin{tabular}{|c|c|c|}
\hline
Funktion & Potenzreihe & Konvergenzradius \\
\hline
\( e^x \) & \( \sum\limits_{n=0}^\infty \frac{x^n}{n!} \) & \( \infty \) \\
\hline
\( \sin(x) \) & \( \sum\limits_{n=0}^\infty \frac{(-1)^n x^{2n+1}}{(2n+1)!} \) & \( \infty \) \\
\hline
\( \cos(x) \) & \( \sum\limits_{n=0}^\infty \frac{(-1)^n x^{2n}}{(2n)!} \) & \( \infty \) \\
\hline
\( \frac{1}{1 - x} \) & \( \sum\limits_{n=0}^\infty x^n \) & \( 1 \) \\
\hline
\( \ln(1 + x) \) & \( \sum\limits_{n=1}^\infty \frac{(-1)^{n+1} x^n}{n} \) & \( 1 \) \\
\hline
\( \arctan(x) \) & \( \sum\limits_{n=0}^\infty \frac{(-1)^n x^{2n+1}}{2n+1} \) & \( 1 \) \\
\hline
\end{tabular}



% Stetigkeit und Topologie___________________________________________________________________________________________________
%%%%%%
\subsection{4. Stetigkeit, Topologie}
\textbf{Stetigkeit in } $x_0 \in D \subset \mathbb{R}$:  
\[
\forall \varepsilon > 0\; \exists \delta > 0:\; |x - x_0| < \delta \Rightarrow |f(x) - f(x_0)| < \varepsilon
\]

\textbf{f ist stetig in } D, falls f in jedem Punkt von D stetig ist.

\textbf{Grenzwert in } $x_0$:  
\[
\lim_{x \to x_0} f(x) = y_0 \Leftrightarrow
\forall \varepsilon > 0\; \exists \delta > 0:\; |x - x_0| < \delta \Rightarrow |f(x) - y_0| < \varepsilon
\]

\textbf{Beispiel:}  
$f(x) = 3x - 2 \quad \Rightarrow \lim_{x \to a} f(x) = 3a - 2$

\textbf{Zwischenwertsatz:}  
f stetig auf [a, b],\; f(a) \leq f(b),\; c \in [f(a), f(b)]  
\[
\Rightarrow \exists \bar{x} \in [a, b]: f(\bar{x}) = c
\]

\textbf{Verknüpfung:}  
f stetig in \( x_0 \),\; g stetig in \( f(x_0) \)  
\[
\Rightarrow g \circ f \text{ stetig in } x_0
\]

\textbf{Grenzwert:}  
\[
\lim_{x \to x_0} f(x) = y_0 \Leftrightarrow
\forall \varepsilon > 0\; \exists \delta > 0:\;
\|x - x_0\| \leq \delta \Rightarrow \|f(x) - y_0\| \leq \varepsilon
\]

\textbf{Schreibweise:}  
\( \lim\limits_{x \to x_0} f(x) = y_0 \)

\textbf{Kompaktheit:}  
\( A \subset \mathbb{R}^n \) ist kompakt  
\[
\Leftrightarrow \text{abgeschlossen und beschränkt}
\]
\[
\Leftrightarrow \text{jede Folge in } A \text{ hat konvergente Teilfolge in } A
\]

\textbf{Satz:}  
Ist \( f: [a,b] \to \mathbb{R} \) stetig, dann ist \( f \) beschränkt.

\textbf{Beispiel:}  
\( f(x) = \frac{1}{1 + x^2} \) auf \( [-2, 2] \) stetig ⇒ beschränkt durch \( 0 < f(x) \leq 1 \)

\textbf{Satz:}  
\( K \subset \mathbb{R}^n \) kompakt,\; \( f: K \to \mathbb{R}^m \) stetig  
\[
\Rightarrow f(K) \text{ ist kompakt}
\]


\textbf{Korollar:}  
\( f: K \to \mathbb{R} \) stetig, \( K \subset \mathbb{R}^n \) kompakt, nicht leer  
\[
\Rightarrow f \text{ besitzt Maximum und Minimum auf } K
\]


\textbf{Lemma:}  
Für nichtleere kompakte \( Q \subset \mathbb{R} \):  
\[
\sup Q, \inf Q \in Q
\]

% Differentialrechnung auf R___________________________________________________________________________________________________
%%%%%%
\subsection{5. Differentialrechnung auf R}

% Integration___________________________________________________________________________________________________
%%%%%%
\subsection{6. Integration}