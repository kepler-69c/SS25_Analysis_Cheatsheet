\section{Analysis I - draft}

% Grundlagen und Logik___________________________________________________________________________________________________
%%%%%%
\subsection{1. Grundlagen und Logik}
\textbf{Aussagen} sind wahr/falsch.  
Negation: $\lnot A$.  

Konjunktion: $A\wedge B$.  

Disjunktion: $A\vee B$.  

Implikation: $A\Rightarrow B$ (wahr, ausser $A=T,B=F$).  

Äquivalenz: $A\Leftrightarrow B$ ($A\Rightarrow B$ und $B\Rightarrow A$). 

\textbf{Modus ponens}: Aus $A$ und $A\Rightarrow B$ folgt $B$.  

\textbf{Kontraposition}: $A\Rightarrow B\equiv(\lnot B)\Rightarrow(\lnot A)$.  

\textbf{Widerspruchsbeweis}: Annahme $\lnot A\Rightarrow\,$Widerspruch.  

\textbf{Induktion}:  
\[
\Bigl(P(0)\;\wedge\;\forall k:\,P(k)\Rightarrow P(k+1)\Bigr)\;\Rightarrow\;\forall n:P(n).
\]

\textbf{Satz:} \(\displaystyle \sum_{i=1}^n i = \frac{n(n+1)}{2} \quad \forall n \in \mathbb{N}_0\)

\textbf{Induktion:}

Für n = 0 gilt
IA: \(n = 0\): \(0 = \frac{0(0+1)}{2}\;\checkmark\)

Es sei $k \in \mathbb{N}$ beliebig, und es gelte

IV: \(\sum_{i=1}^k i = \frac{k(k+1)}{2}\)

Dann gilt die Aussage auch für k + 1:

IS: \(\sum_{i=1}^{k+1} i = \sum_{i=1}^k i + (k+1)
= \frac{k(k+1)}{2} + (k+1)
= \frac{(k+1)(k+2)}{2}\;\checkmark\)

Sei \( f : X \to Y \) eine Abbildung.

\begin{itemize}[left=0pt, itemsep=1.5em]

  \item \textbf{Surjektiv:} \( f \) heisst surjektiv, falls jedes \( y \in Y \) mindestens ein Urbild hat, d.\,h.
  \[
  \forall y \in Y\ \exists x \in X : f(x) = y
  \]

  \item \textbf{Injektiv:} \( f \) heisst injektiv, falls jedes \( y \in Y \) höchstens ein Urbild hat, d.\,h.
  \[
  \forall x_1, x_2 \in X : f(x_1) = f(x_2) \Rightarrow x_1 = x_2
  \]

  \item \textbf{Bijektiv:} \( f \) heißt bijektiv, falls jedes \( y \in Y \) genau ein Urbild hat,\\
  d.\,h. wenn \( f \) sowohl surjektiv als auch injektiv ist.

\end{itemize}

% Zahlen und Vektoren___________________________________________________________________________________________________
%%%%%%
\subsection{2. Zahlen und Vektoren}
\textbf{N, Z, Q, R, C.}  
\textbf{Reelle Zahlen:} vollständiges geordnetes Feld.  
\textbf{Supremum/Infimum:} Für $A\subset\R$ beschränkt:  
\begin{align*}
\sup A:=\text{kleinstes obere Schranke},\\
\inf A:=\text{grösste untere Schranke}.
\end{align*}
\textbf{Komplexe Zahlen:} $z=a+ib,\;a,b\in\R,\;i^2=-1$.  
Betrag: $|z|=\sqrt{a^2+b^2}$.  
Konjugierte: $\overline z=a-ib$.  
\begin{align*}
z_1 + z_2 &= (a_1 + a_2) + i(b_1 + b_2) \\
z_1 z_2 &= (a_1 a_2 - b_1 b_2) + i(a_1 b_2 + a_2 b_1)
\end{align*}
Polarform: $z=r e^{i\varphi},\;r=|z|,\;\varphi=\arg z$.

% Folgen und Reihen ___________________________________________________________________________________________________
%%%%%%
\subsection{3. Folgen und Reihen}  
Eine \emph{ Zahlenfolge} ist eine Funktion  
\[
a : \mathbb{N}_0 \to \mathbb{C}, \quad n \mapsto a_n := a(n).
\]  

\textbf{Konvergenz einer Folge:}  
Eine Folge \( (x_n) \subset \mathbb{R} \) \textbf{konvergiert} gegen \( A \in \mathbb{R} \), falls  
\[
\forall \varepsilon > 0\; \exists N \in \mathbb{N}:\; \forall n \geq N:\; |x_n - A| < \varepsilon.
\]  
Dann schreiben wir \( \lim_{n \to \infty} x_n = A \) und nennen \( A \) den \textbf{Grenzwert}.

\[
Konvergenz \to Beschränktheit
\]  

\textbf{Beispiel Konvergent:} \( x_n = \frac{1}{n} \Rightarrow \lim\limits_{n \to \infty} x_n = 0 \)

\textbf{Beispiel Divergent:} \( x_n = n \Rightarrow \lim\limits_{n \to \infty} x_n = \infty \)

\textbf{Rechenregeln für konvergente Folgen:}  
Seien \( \lim\limits_{n \to \infty} x_n = A \), \( \lim\limits_{n \to \infty} y_n = B \), \( \alpha \in \mathbb{R} \):

\begin{itemize}
  \item \( \lim\limits_{n \to \infty} (x_n + y_n) = A + B \)
  \item \( \lim\limits_{n \to \infty} (x_n y_n) = AB \)
  \item \( \lim\limits_{n \to \infty} (\alpha x_n) = \alpha A \)
  \item Falls \( x_n \neq 0\ \forall n \), \( A \neq 0 \):  
        \( \lim\limits_{n \to \infty} \left( \frac{1}{x_n} \right) = \frac{1}{A} \)
  \item Falls \( x_n \leq y_n\ \forall n \): \( A \leq B \)
\end{itemize}

\textbf{Monotone Konvergenz:}  
Eine monotone Folge \( (x_n) \subset \mathbb{R} \) konvergiert genau dann, wenn sie beschränkt ist.

\begin{itemize}
  \item Monoton wachsend:  
  \( \lim\limits_{n \to \infty} x_n = \sup \{ x_n \mid n \in \mathbb{N} \} \)

  \item Monoton fallend:  
  \( \lim\limits_{n \to \infty} x_n = \inf \{ x_n \mid n \in \mathbb{N} \} \)
\end{itemize}

\textbf{Euler'sche Zahl:}  
\[
e := \lim\limits_{n \to \infty} \left( 1 + \frac{1}{n} \right)^n
\]

\textbf{Eigenschaften:}
\begin{itemize}
  \item \( e \) ist wohldefiniert (Folge konvergiert).
  \item \( e \approx 2{,}718\ldots \) (irrational, nicht periodisch).
\end{itemize}

\textbf{Obere/Untere Grenzwerte:}  
Für eine beschränkte Folge \( (x_n) \subset \mathbb{R} \) gilt:

\[
\limsup\limits_{n \to \infty} x_n = \lim\limits_{n \to \infty} \left( \sup_{k \geq n} x_k \right)
\]
\[
\liminf\limits_{n \to \infty} x_n = \lim\limits_{n \to \infty} \left( \inf_{k \geq n} x_k \right)
\]

\textbf{Immer:}  
\[
\liminf x_n \leq \limsup x_n
\]
\textbf{Bernoulli-de l'Hôspital:} (Abl. existiert, f und g diffbar um a und $g'(x) \not = 0$)
\[
\lim\limits_{x \to a} \frac{f(x)}{g(x)} = \lim\limits_{x \to a} \frac{f'(x)}{g'(x)}
\]
\textbf{\( e^{\log} \)–Trick:}  
\[
\lim f(x)^{g(x)} = \exp\left( \lim g(x) \cdot \log f(x) \right)
\quad \text{falls } f(x) > 0
\]

\textbf{Cauchy-Folge:}  
$(x_n) \subset \mathbb{R}$ ist Cauchy  
$\Leftrightarrow \forall \varepsilon > 0\; \exists N \in \mathbb{N}:\; \forall m,n \geq N:\; |x_n - x_m| < \varepsilon$

\textbf{Satz:}  
$
(x_n) \text{ konvergiert } \Leftrightarrow \text{Cauchy-Folge}
$

\textbf{Unendliche Grenzwerte:}  

$
\lim_{n \to \infty} x_n = \infty \Leftrightarrow  
\forall M > 0\; \exists N \in \mathbb{N}:\; x_n > M\; \forall n \geq N
$

$
\lim_{n \to \infty} x_n = -\infty \Leftrightarrow  
\forall M > 0\; \exists N \in \mathbb{N}:\; x_n < -M\; \forall n \geq N
$

\textbf{Reihe }
$
\sum_{k=0}^\infty a_k \text{ konvergiert} \Leftrightarrow \newline
\lim_{n \to \infty} s_n = A \in \mathbb{R},\quad s_n := \sum_{k=0}^n a_k
$
\[
\textbf{Wenn } \sum a_k \text{ konvergiert, dann } \lim a_k = 0
\]

\textbf{Restlemma:}  
\[
\sum_{k=0}^\infty a_k \text{ konvergiert } \Leftrightarrow  
\sum_{k=N}^\infty a_k \text{ konvergiert } \quad \forall N \in \mathbb{N}
\]
\footnotesize
\renewcommand{\arraystretch}{1.3}

\begin{center}
\begin{tabular}{|>{\centering\arraybackslash}p{1.4cm}|
                >{\centering\arraybackslash}p{1.7cm}|
                >{\centering\arraybackslash}p{2.4cm}|}
\hline
\textbf{Objekt} & \textbf{Bedingung} & \textbf{Konvergenz / Wert} \\
\hline
$a_k = \frac{1}{k}$ 
  & — 
  & konvergent, $\lim\limits_{k\to\infty} a_k = 0$ \\
\hline
$a_k = z^k$ 
  & $|z|<1$ 
  & konvergent, $\lim\limits_{k\to\infty} z^k = 0$ \\
  & $|z|\ge 1$ 
  & divergent \\
\hline
$\sum\limits_{k=1}^\infty \frac{1}{k}$ 
  & — 
  & divergent (harmonisch) \\
\hline
$\sum\limits_{k=1}^\infty \frac{1}{k^\alpha}$ 
  & $\alpha > 1$ 
  & konvergent \\
  & $\alpha \le 1$ 
  & divergent \\
\hline
$\sum\limits_{k=0}^\infty z^k$ 
  & $|z|<1$ 
  & konvergent, $\sum\limits_{k=0}^\infty z^k = \frac{1}{1-z}$ \\
  & $|z|\ge 1$ 
  & divergent \\
\hline
$\sum\limits_{k=1}^\infty \frac{(-1)^{k-1}}{k}$ 
  & — 
  & konvergent, $\sum\limits \frac{(-1)^{k-1}}{k} = \ln(2)$ \\
\hline
\end{tabular}
\end{center}


\textbf{Majorantenkriterium:}  
\[
0 \leq a_k \leq b_k,\; \sum b_k \text{ konv.} \;\Rightarrow\; \sum a_k \text{ konv.}
\]

\textbf{Minorantenkriterium:}  
\[
0 \leq a_k \leq b_k,\; \sum a_k \text{ divergiert} \;\Rightarrow\; \sum b_k \text{ divergiert}
\]

\textbf{Cauchy-Kondensation:}  
\[
(a_k) \downarrow,\; a_k \geq 0:  
\quad \sum a_k \text{ konv.} \;\Leftrightarrow\; \sum 2^k a_{2^k} \text{ konv.}
\]

\textbf{Leibniz-Kriterium (alternierende Reihen)}  
Anwendbar für Reihen der Form:
\[
\sum_{k=0}^\infty (-1)^k a_k \quad \text{oder} \quad \sum_{k=0}^\infty (-1)^{k+1} a_k
\]

\textbf{Leibniz-Kriterium:}  
\(
\sum (-1)^k a_k \text{ konvergiert, falls:}
\)

\begin{tabular}{ll}
\checkmark & \( a_k \geq 0 \) \\
\checkmark & \( a_k \downarrow \) (monoton fallend) \\
\checkmark & \( \lim a_k = 0 \)
\end{tabular}

\vspace{0.5cm}

\textbf{Beispiel: } \( \sum_{k=1}^\infty \frac{(-1)^k}{k} \)

\begin{tabular}{ll}
\checkmark & \( \frac{1}{k} \geq 0 \) \\
\checkmark & \( \frac{1}{k+1} < \frac{1}{k} \) or \( f(x) = \frac{1}{x},\ f'(x) = -\frac{1}{x^2} < 0 \) ⇒ mon. fall.\\
\checkmark & \( \lim \frac{1}{k} = 0 \)
\Rightarrow \text{ konvergiert nach Leibniz}
\end{tabular}

\textbf{Wurzelkriterium:}  
\( \alpha := \limsup \sqrt[n]{|a_n|} \)

$
\alpha < 1 \Rightarrow \sum a_n \text{ konv. absolut}, \newline
\alpha > 1 \Rightarrow \sum a_n \text{ divergiert}
$

\textbf{Beispiel: } \( a_n = \frac{1}{3^n} \Rightarrow \sqrt[n]{|a_n|} = \frac{1}{3} \Rightarrow \alpha = \frac{1}{3} < 1 \)

\( \Rightarrow \sum a_n \) konv. abs.

\textbf{Quotientenkriterium:}  
\( \alpha := \lim \left| \frac{a_{n+1}}{a_n} \right| \)

\[
\alpha < 1 \Rightarrow \sum a_n \text{ konv. absolut}, \quad
\alpha > 1 \Rightarrow \sum a_n \text{ divergiert}
\]

\textbf{Beispiel: } \( a_k = \frac{(k!)^2}{(2k)!} \)

\[
\frac{a_{k+1}}{a_k} = \frac{((k+1)!)^2}{(2k+2)!} \cdot \frac{(2k)!}{(k!)^2}
= \frac{(k+1)^2 \cdot (k!)^2}{(2k+2)(2k+1)(2k)!} \cdot \frac{(2k)!}{(k!)^2}
= \frac{(k+1)^2}{(2k+2)(2k+1)}
\]

\[
\Rightarrow \lim \frac{(k+1)^2}{(2k+2)(2k+1)} = \frac{1}{4} < 1
\Rightarrow \sum a_k \text{ konv. abs.}
\]

\textbf{Zeta-Reihe:}  
\[
\zeta(s) := \sum_{n=1}^\infty \frac{1}{n^s}, \quad s > 1
\]

\textbf{Produktsatz:}  
Wenn \( \sum a_n \), \( \sum b_n \) absolut konvergieren, dann konvergiert auch  
\[
\sum_{n=0}^\infty a_{\alpha_1(n)} b_{\alpha_2(n)}
= \left( \sum a_n \right)\left( \sum b_n \right)
\]
für jede Bijektion \( \alpha: \mathbb{N} \to \mathbb{N} \times \mathbb{N} \)

\textbf{Cauchy-Produkt:}  
Wenn \( \sum a_n \), \( \sum b_n \) absolut konvergieren, dann:
\[
\left( \sum a_n \right) \left( \sum b_n \right)
= \sum_{n=0}^\infty \left( \sum_{k=0}^n a_k b_{n-k} \right)
\]

\textbf{Potenzreihe:}  
\[
\sum_{n=0}^\infty a_n x^n
\]

\textbf{Konvergenzradius:}  
Gegeben Potenzreihe \( \sum a_n x^n \), setze
\[
\rho = \limsup_{n \to \infty} \sqrt[n]{|a_n|} \quad \text{or} \quad
\rho = \limsup_{n \to \infty} \left| \frac{a_{n+1}}{a_n} \right|,
R = \frac{1}{\rho}
\]

\[
R =
\begin{cases}
0, & \text{falls } \rho = \infty \\
\infty, & \text{falls } \rho = 0 \\
\rho^{-1}, & \text{sonst}
\end{cases}
\]

\textbf{Verhalten:}
\begin{itemize}
  \item \( |x| < R \) \quad absolut konvergent
  \item \( |x| > R \) \quad divergent
  \item \( |x| = R \) \quad einzeln prüfen
\end{itemize}

\vspace{0.5em}

\textbf{Summe und Produkt von Potenzreihen:}  
Falls \( \sum a_n x^n \), \( \sum b_n x^n \) Konvergenzradius \( \geq R \) haben, dann gilt:
\[
\sum (a_n + b_n)x^n, \quad
\sum_{n=0}^\infty \left( \sum_{k=0}^n a_k b_{n-k} \right) x^n
\quad \text{haben Radius } \geq R
\]

\vspace{0.5em}

\textbf{Exponentialfunktion:}
\[
e^x = \sum_{k=0}^\infty \frac{x^k}{k!} = \lim_{n \to \infty} \left(1 + \frac{x}{n}\right)^n
\quad \text{mit Radius } R = \infty
\]

\textbf{Eulerformel:}  
\[
e^{ix} = \cos(x) + i \sin(x)
\Rightarrow \\
\sin(x) = \frac{e^{ix} - e^{-ix}}{2i}, \quad
\cos(x) = \frac{e^{ix} + e^{-ix}}{2}
\]

\textbf{Additionstheoreme:}  
\[
\sin(x + y) = \sin x \cos y + \cos x \sin y
\]
\[
\cos(x + y) = \cos x \cos y - \sin x \sin y
\]

\textbf{Periodizität:}
\[
\begin{aligned}
\sin(x + \tfrac{\pi}{2}) &= \cos(x) & \qquad \cos(x + \tfrac{\pi}{2}) &= -\sin(x) \\
\sin(x + \pi) &= -\sin(x) & \cos(x + \pi) &= -\cos(x) \\
\sin(x + 2\pi) &= \sin(x) & \cos(x + 2\pi) &= \cos(x)
\end{aligned}
\]

\textbf{Wichtige Potenzreihen:}
\begin{tabular}{|c|c|c|}
\hline
Funktion & Potenzreihe & Konvergenzradius \\
\hline
\( e^x \) & \( \sum\limits_{n=0}^\infty \frac{x^n}{n!} \) & \( \infty \) \\
\hline
\( \sin(x) \) & \( \sum\limits_{n=0}^\infty \frac{(-1)^n x^{2n+1}}{(2n+1)!} \) & \( \infty \) \\
\hline
\( \cos(x) \) & \( \sum\limits_{n=0}^\infty \frac{(-1)^n x^{2n}}{(2n)!} \) & \( \infty \) \\
\hline
\( \frac{1}{1 - x} \) & \( \sum\limits_{n=0}^\infty x^n \) & \( 1 \) \\
\hline
\( \ln(1 + x) \) & \( \sum\limits_{n=1}^\infty \frac{(-1)^{n+1} x^n}{n} \) & \( 1 \) \\
\hline
\( \arctan(x) \) & \( \sum\limits_{n=0}^\infty \frac{(-1)^n x^{2n+1}}{2n+1} \) & \( 1 \) \\
\hline
\end{tabular}



% Stetigkeit und Topologie___________________________________________________________________________________________________
%%%%%%
\subsection{4. Stetigkeit, Topologie}
\textbf{Stetigkeit in } $x_0 \in D \subset \mathbb{R}$:  
\[
\forall \varepsilon > 0\; \exists \delta > 0:\; |x - x_0| < \delta \Rightarrow |f(x) - f(x_0)| < \varepsilon
\]

\textbf{f ist stetig in } D, falls f in jedem Punkt von D stetig ist.

\textbf{Gleichmässige Stetigkeit}\\
Eine Funktion heisst ($\delta$ nicht abhängig von $x_0$) \uline{gleichmässig stetig} auf $\Omega$, falls gilt
\[
\forall \varepsilon > 0 \; \exists \delta > 0 \; \forall x, x_0 \in \Omega : |x - x_0| < \delta \Rightarrow |f(x) - f(x_0)| < \varepsilon
\]

\textbf{Beispiele stetiger/unstetiger Funktionen:}
\begin{itemize}
  \item[(ii)] \textbf{Identität:} $\mathrm{id}_S(x) = x$ ist stetig für $S \subset \mathbb{R}^n$  
  Beweis: $\|\mathrm{id}(x) - \mathrm{id}(x_0)\| = \|x - x_0\| \leq \delta = \varepsilon$

  \item[(iii)] \textbf{Projektion:} $\mathrm{pr}_i : \mathbb{R}^n \to \mathbb{R},\; \mathrm{pr}_i(x) = x_i$ ist stetig
%
%   \item[(iv)] \textbf{Indikatorfunktion:}  
%     \[
%       \chi_A(x) = \begin{cases}
%         1, & x \in A \\
%         0, & \text{sonst}
%       \end{cases}
%     \quad\text{nicht stetig an Randpunkten von } A
%     \]
%     Bsp: $\chi_{\{0\}}$ unstetig in $x = 0$  
%     \[
%       \delta = \tfrac{1}{2},\quad |x - 0| \leq \delta \Rightarrow
%       |\chi_{\{0\}}(x) - \chi_{\{0\}}(0)| = 1 > \varepsilon = \tfrac{1}{2}
%     \]

%   \item[(v)] \textbf{Dirichlet-Funktion:} $\chi_{\mathbb{Q}}:\mathbb{R} \to \mathbb{R}$,  
%     \[
%       \chi_{\mathbb{Q}}(x) = \begin{cases}
%         1, & x \in \mathbb{Q} \\
%         0, & \text{sonst}
%       \end{cases}
%     \]
%     ist in keinem Punkt stetig (Sprungfunktion aller Dichten).
% \end{itemize}
%
\textbf{Stetigkeit \& Rechenoperationen:}
\begin{itemize}
  \item Addition, Subtraktion, Multiplikation, Division (ausser bei Nenner \(=0\)) komplexer Zahlen sind stetig.

  \item Seien \(f,g : S \subset \mathbb{R}^n \to \mathbb{C}\), \(a \in \mathbb{C}\), \(x_0 \in S\), \(f,g\) stetig in \(x_0\). Dann sind auch stetig in \(x_0\):  
  \[
  f + g,\quad a \cdot f,\quad f \cdot g
  \]
  Falls \(g(x_0) \ne 0\): auch \(\frac{f}{g}\) stetig in \(x_0\) (auf \(S \cap \{x \mid g(x) \ne 0\}\)).

  \item Seien \(f = (f_1, \dots, f_{n'}) : S \to \mathbb{R}^{n'}\). Dann ist  
  \[
  f \text{ stetig in } x_0 \iff \text{alle } f_i \text{ stetig in } x_0.
  \]
\end{itemize}

\textbf{Stetigkeit von Polynomen:}
\begin{itemize}
  \item[(i)] \(f(x) = a_1 x + a_0\) mit \(a_0,a_1 \in \mathbb{R}\) ist stetig (Summe konst. und linearer Funktion).
  \item[(ii)] Jedes reelle Polynom \(p: \mathbb{R} \to \mathbb{R}\) ist stetig.
  \item[(iii)] Jedes komplexe Polynom \(p: \mathbb{C} \to \mathbb{C}\) ist stetig.
\end{itemize}

\textbf{Wurzelfunktion stetig:}  
Für alle \(n \in \mathbb{N}\) ist \(\sqrt[n]{\cdot} : [0,\infty) \to [0,\infty)\) stetig.

\textbf{Durch Potenzfunktion definierte Funktion ist stetig:}  
Funktion \(f(z) = \sum_{k=0}^\infty c_k z^k\) ist auf der Konvergenzscheibe stetig.

\textbf{Bemerkung:}  
Die \textit{Konvergenzscheibe} ist die grösste offene Kreisscheibe, auf der die Potenzreihe konvergiert. Auf ihr ist \(f\) stetig.

\textbf{Innerer Punkt \& Inneres:}  
\(x \in S\) ist innerer Punkt \(\Leftrightarrow \exists r>0: B_r^n(x) \subseteq S\)

\[
\text{Int}(S) := \{x \in S \mid x \text{ innerer Punkt von } S\}
\]

\textbf{Vereinigung:} \(\displaystyle \bigcup S := \{x \mid \exists s \in S: x \in s\}\)

\textbf{Durchschnitt:} \(\displaystyle \bigcap S := \{x \mid \forall s \in S: x \in s\}\quad (S \ne \emptyset)\)


\textbf{Eigenschaften offener Mengen:}
\begin{itemize}
  \item[(i)] \(\emptyset\) und \(\mathbb{R}^n\) sind offen
  \item[(ii)] Endlicher Durchschnitt offener Mengen ist offen
  \item[(iii)] Beliebige Vereinigung offener Mengen ist offen
\end{itemize}


\textbf{Offenheit:}  
\(S \subset \mathbb{R}^n\) ist offen \(\Leftrightarrow S = \text{Int}(S)\)

\textbf{Merkmale:}
\begin{itemize}
  \item \(\text{Int}(S) \subseteq S\)
  \item offen \(\Leftrightarrow\) gleich ihrem Inneren
\end{itemize}

\textbf{Abgeschlossenheit:}  
\(A \subset \mathbb{R}^n\) ist abgeschlossen \(\Leftrightarrow \mathbb{R}^n \setminus A\) ist offen.

\textbf{Eigenschaften abgeschlossener Mengen:}
\begin{itemize}
  \item[(i)] \(\emptyset,\, \mathbb{R}^n\) sind abgeschlossen
  \item[(ii)] Endliche Vereinigungen abgeschlossener Mengen sind abgeschlossen
  \item[(iii)] Beliebige Durchschnitte abgeschlossener Mengen sind abgeschlossen
\end{itemize}

\textbf{Folgenkriterium:}  
\(S \subset \mathbb{R}^n\) ist abgeschlossen \(\Leftrightarrow\)  
\(\forall (x_k) \subset S,\; x_k \to x \Rightarrow x \in S\)

\textbf{Abschluss:}  
\[
\overline{S} := \bigcap \{A \subset \mathbb{R}^n \mid A \text{ abgeschlossen, } S \subset A\}
\]

\textbf{Eigenschaften:}
\begin{itemize}
  \item \(\overline{S} \supseteq S\)
  \item \(\overline{S}\) ist abgeschlossen
  \item \(\overline{S}\) ist kleinste abgeschlossene Menge mit \(S \subseteq \overline{S}\)
  \item \(S\) ist genau dann abgeschlossen, wenn \(S = \overline{S}\)
\end{itemize}

\textbf{Satz (Inneres \& Abschluss):}
\begin{itemize}
  \item[(i)] \(\displaystyle \text{Int}(S) = \bigcup_{U \subseteq S,\; U \text{ offen}} U\)
  \item[(ii)] \(\displaystyle \overline{S} = \{x \in \mathbb{R}^n \mid \exists (x_k) \subset S,\; x_k \to x\}\)
\end{itemize}

\textbf{Rand:}  
\(\partial S := \overline{S} \setminus \text{Int}(S)\)

\textbf{Rand ist abgeschlossen:}  
\(\partial S = \overline{S} \setminus \text{Int}(S) = \overline{S} \cap (\mathbb{R}^n \setminus S)\)

\textbf{Satz (Randcharakterisierung):}  
\[
\partial S = \{x \in \mathbb{R}^n \mid \forall r > 0: \\
\; B_r(x) \cap S \ne \emptyset \;\land\; B_r(x) \cap (\mathbb{R}^n \setminus S) \ne \emptyset \}
\]


\textbf{Grenzwert in } $x_0$:  
\[
\lim_{x \to x_0} f(x) = y_0 \Leftrightarrow
\forall \varepsilon > 0\; \exists \delta > 0: \\
\; |x - x_0| < \delta \Rightarrow |f(x) - y_0| < \varepsilon
\]

\textbf{Beispiel:}  
$f(x) = 3x - 2 \quad \Rightarrow \lim_{x \to a} f(x) = 3a - 2$

\textbf{Varia für Mengen}
\renewcommand{\arraystretch}{1.2}

\begin{center}
\begin{tabularx}{\linewidth}{l 
  >{\centering\arraybackslash}p{0.5cm} 
  >{\centering\arraybackslash}p{0.1cm} 
  >{\centering\arraybackslash}p{0.2cm} 
  >{\centering\arraybackslash}p{0.2cm} 
  >{\centering\arraybackslash}p{0.3cm} 
  >{\centering\arraybackslash}p{0.3cm} 
  >{\centering\arraybackslash}p{0.3cm}}
\toprule
& $B_r(x_0)$ & $\emptyset$ & $a$ & $\mathbb{R}$ & $(a,b]$ & $[a,\infty)$ & $(a,\infty)$ \\
\midrule
offen     & Ja   & Ja  & Nein & Ja  & Nein & Nein & Ja \\
abg. & Nein & Ja  & Ja   & Ja  & Nein & Ja   & Nein \\
\bottomrule
\end{tabularx}
\end{center}
\begin{itemize}[leftmargin=1.5em]
  \item Das innere einer Menge ist immer offen
  \item $\Omega$ offen $\Rightarrow \mathring{\Omega} = \Omega$
  \item Falls die Menge nur aus isolierten Punkten besteht (z.B. $\mathbb{Q}$), dann gilt $\mathring{\Omega} = \emptyset$
  \item Der Abschluss einer Menge ist immer abgeschlossen
  \item $\Omega$ abgeschlossen $\Rightarrow \overline{\Omega} = \Omega$
  \item $(A \cap B)^\circ = \mathring{A} \cap \mathring{B}$ \quad und \quad $\overline{A \cup B} = \overline{A} \cup \overline{B}$
\end{itemize}

\textbf{Zwischenwertsatz:}  
f stetig auf [a, b],\; f(a) \leq f(b),\; c \in [f(a), f(b)]  
\[
\Rightarrow \exists \bar{x} \in [a, b]: f(\bar{x}) = c
\]

\textbf{Verknüpfung:}  
f stetig in \( x_0 \),\; g stetig in \( f(x_0) \)  
\[
\Rightarrow g \circ f \text{ stetig in } x_0
\]

\textbf{Monotonie:}
\begin{itemize}
  \item \(f\) monoton wachsend \(\Leftrightarrow x \le x' \Rightarrow f(x) \le f(x')\)
  \item \(f\) streng monoton wachsend \(\Leftrightarrow x < x' \Rightarrow f(x) < f(x')\)
\end{itemize}

\textbf{Strenge Monotonie $\Rightarrow$ Injektivität}

\textbf{k-te Wurzel:}
\begin{itemize}
  \item \(k\) gerade: \(\sqrt[k]{\cdot} := p_k^{-1}: [0, \infty) \to [0, \infty)\)
  \item \(k\) ungerade: \(\sqrt[k]{\cdot} := p_k^{-1}: \mathbb{R} \to \mathbb{R}\)
\end{itemize}

\textbf{Logarithmus:}  
\(\log := \exp^{-1} : (0,\infty) \to \mathbb{R}\)

\textbf{Produktregel:}  
\(\log(xy) = \log(x) + \log(y)\) für \(x, y > 0\)

\textbf{Satz (Stetige Umkehrfunktion):}  
\(f: K \to Y\) stetig und bijektiv,\; \(K\) kompakt  
\(\Rightarrow f^{-1}: Y \to K\) ist stetig

\textbf{Satz (Bild und Umkehrfunktion bei Monotonie):}  
Sei \(f: I \to \mathbb{R}\) stetig auf Intervall \(I\), dann gilt:
\begin{itemize}
  \item[(i)] \(\text{Bild}(f)\) ist ein Intervall
  \item[(ii)] Falls \(f\) streng monoton wachsend, dann ist \(f^{-1}: \text{im}(f) \to I\) stetig und streng monoton wachsend
\end{itemize}

\textbf{Satz (stetige Umkehrfunktion bei offenem \(U\)):}  
Sei \(f: U \to \mathbb{R}^p\) stetig und injektiv, \(U \subset \mathbb{R}^n\) offen, \(U \ne \emptyset\), \(n \ge p\)

\begin{itemize}
  \item[(i)] \(n = p\)
  \item[(ii)] \(f(U)\) ist offen in \(\mathbb{R}^n\)
  \item[(iii)] \(f^{-1}: \text{im}(f) \to U\) ist stetig
\end{itemize}


\textbf{Punktweise Konvergenz:}  
\(f_m \to f\) punktweise auf \(X\)  
\(\Leftrightarrow \forall x \in X:\; f_m(x) \to f(x)\)

Die Folge $(f_m)_{m \in \mathbb{N}_0}$ konvergiert punktweise gegen $f$ auf $X$ genau dann, wenn gilt:
\[
\forall x \in X\; \forall \varepsilon > 0\; \exists m_0 \in \mathbb{N}_0\; \forall m \ge m_0:\; \|f_m(x) - f(x)\| \le \varepsilon
\]

\textbf{Gleichmässige Konvergenz:}  
\(f_m \to f\) gleichmässig auf \(X\)  
\(\Leftrightarrow \sup_{x \in X} \|f_m(x) - f(x)\| \to 0\)

Äquivalent:  
\(\forall \varepsilon > 0\; \exists m_0 \in \mathbb{N}:\; \forall m \ge m_0,\; \forall x \in X:\; \|f_m(x) - f(x)\| \le \varepsilon\)

\textbf{Beispiel:} \(f_m(x) = x^m\) auf \([0,1] \Rightarrow\) punktweise Konvergenz gegen  
\[
f(x) = \begin{cases}
0, & x < 1 \\
1, & x = 1
\end{cases}
\quad \text{(nicht gleichmässig)}
\]

\textbf{Satz (Stetigkeit bei gleichmässiger Konvergenz):}  
\(f_m\) stetig,\; \(f_m \to f\) gleichmässig \(\Rightarrow f\) ist stetig


\textbf{Grenzwert:}  
\[
\lim_{x \to x_0} f(x) = y_0 \Leftrightarrow
\forall \varepsilon > 0\; \exists \delta > 0:\;
\|x - x_0\| \leq \delta \Rightarrow \|f(x) - y_0\| \leq \varepsilon
\]

\textbf{Schreibweise:}  
\( \lim\limits_{x \to x_0} f(x) = y_0 \)

\textbf{Beschränktheit:}  
\(A \subset \mathbb{R}^n\) ist beschränkt \(\Leftrightarrow \exists R > 0, x_0 \in \mathbb{R}^n:\; A \subset \overline{B}_R(x_0)\)

\textbf{Kompaktheit:}  
\[
A \subset \mathbb{R}^n \text{ ist kompakt } \Leftrightarrow A \text{ ist abgeschlossen und beschränkt}
\]
\[
\Leftrightarrow \text{jede Folge in } A \text{ hat konvergente Teilfolge in } A
\]

\textbf{Satz:}  
\(K \subset \mathbb{R}^n\) kompakt,\; \(f: K \to \mathbb{R}^p\) stetig  
\(\Rightarrow f(K)\) ist kompakt

\textbf{Korollar (Maximum/Minimum auf Kompaktum):}  
\(K \subset \mathbb{R}^n\) kompakt, \(K \ne \emptyset\), \(f: K \to \mathbb{R}\) stetig  
\(\Rightarrow f\) hat Maximum und Minimum auf \(K\)

\textbf{Satz (Stetigkeit via Urbild):}  
Für \(f: X \to Y\) sind äquivalent:
\begin{itemize}
  \item[(a)] \(f\) ist stetig
  \item[(b)] \(f^{-1}(V)\) ist relativ offen in \(X\) für jede relativ offene Menge \(V \subseteq Y\)
  \item[(c)] \(f^{-1}(B)\) ist relativ abgeschlossen in \(X\) für jede relativ abgeschlossene Menge \(B \subseteq Y\)
\end{itemize}

\textbf{Umgebung:}  
\(U \subseteq X\) ist Umgebung von \(x_0 \in X\)  
\(\Leftrightarrow \exists r > 0: B_r^n(x_0) \cap X \subseteq U\)

\textbf{Lemma:}  
\(Q \subset \mathbb{R}\) kompakt, \(Q \ne \emptyset\)  
\(\Rightarrow \inf Q, \sup Q \in Q\)

\textbf{Satz:}  
Ist \( f: [a,b] \to \mathbb{R} \) stetig, dann ist \( f \) beschränkt.

\textbf{Beispiel:}  
\( f(x) = \frac{1}{1 + x^2} \) auf \( [-2, 2] \) stetig ⇒ beschränkt durch \( 0 < f(x) \leq 1 \)

\textbf{Satz:}  
\( K \subset \mathbb{R}^n \) kompakt,\; \( f: K \to \mathbb{R}^m \) stetig  
\[
\Rightarrow f(K) \text{ ist kompakt}
\]


\textbf{Korollar:}  
\( f: K \to \mathbb{R} \) stetig, \( K \subset \mathbb{R}^n \) kompakt, nicht leer  
\[
\Rightarrow f \text{ besitzt Maximum und Minimum auf } K
\]


\textbf{Lemma:}  
Für nichtleere kompakte \( Q \subset \mathbb{R} \):  
\[
\sup Q, \inf Q \in Q
\]

% Differentialrechnung auf R___________________________________________________________________________________________________
%%%%%%
\subsection{5. Differentialrechnung auf R}
\textbf{Ableitung}
\begin{definition}
  Sei $f:I\subset\R\to\R$, $x_0\in I$. Die Ableitung von $f$ in $x_0$ ist
  \[
    f'(x_0) = \lim_{h\to0} \frac{f(x_0 + h) - f(x_0)}{h},
  \]
  falls der Grenzwert existiert. Dann heisst $f$ differenzierbar in $x_0$.
\end{definition}
\begin{itemize}
  \item Differenzierbarkeit impliziert Stetigkeit in $x_0$.
  \item \textbf{Regeln:}
    \[
      (f+g)' = f' + g', \quad 
      (f\cdot g)' = f'\,g + f\,g', \quad
      \Bigl(\tfrac{f}{g}\Bigr)' = \frac{f'\,g - f\,g'}{g^2}.
    \]
  \item \textbf{Kettenregel:} Ist $g$ in $x_0$ differenzierbar, $f$ in $g(x_0)$ differenzierbar, so
    \[
      (f\circ g)'(x_0) = f'(g(x_0)) \cdot g'(x_0).
    \]
    \emph{Beispiel:} $f(u)=u^3$, $g(x)=\sin x$, dann $(f\circ g)'(x)=3\sin^2(x)\cos x$.
  \item \textbf{Potenzregel:} $(x^n)' = n x^{n-1}$ für $n\in\N$.
  \item \textbf{Logarithmus und Exponential:}  
    \[
      (e^x)' = e^x,\quad (\ln x)' = \frac{1}{x},\; x>0.
    \]
  \item \textbf{Umkehrregel:} Ist $f$ bijektiv, differenzierbar und $f'(x)\ne0$, dann
    \[
      (f^{-1})'(y) = \frac{1}{f'(x)},\quad y = f(x).
    \]
    \emph{Beispiel:} $f(x)=e^x$, $f^{-1}(y)=\ln y$, $(\ln y)' = 1/y$.
\end{itemize}

\textbf{Tangente:}  
Wenn \(f\) in \(x_0\) differenzierbar ist, dann ist die Tangente an den Graphen von \(f\) im Punkt \((x_0, f(x_0))\)  
die Gerade durch \((x_0, f(x_0))\) mit Steigung \(f'(x_0)\)

\textbf{Linearität:}  
\(T: V \to W\) ist linear \(\Leftrightarrow\)  
\[
T(av) = aT(v), \quad T(v + v') = T(v) + T(v')\quad \forall a \in \mathbb{R}, v, v' \in V
\]

\textbf{Affinität:}  
\(T: V \to W\) ist affin \(\Leftrightarrow T = \text{linear} + \text{konstant}\)

\textbf{Beispiel:} \(f: \mathbb{R} \to \mathbb{R},\; f(x) = a_1 x + a_0\) ist affin

\textbf{Komponentenweise Differenzierbarkeit:}  
\(f = (f_1, \dots, f_p): \mathbb{R}^n \to \mathbb{R}^p\)  
\(f\) differenzierbar in \(x_0 \Leftrightarrow f_i\) diffbar in \(x_0\) für alle \(i\)  
\[
f'(x_0) = \begin{pmatrix} f_1'(x_0) \\ \vdots \\ f_p'(x_0) \end{pmatrix}
\]

\textbf{Beispiel:} \(f(x) = \begin{pmatrix} x \\ e^x \end{pmatrix} \Rightarrow f'(x_0) = \begin{pmatrix} 1 \\ e^{x_0} \end{pmatrix}\)

\textbf{Satz:}  
Differenzierbarkeit \(\Rightarrow\) Stetigkeit

\textbf{Ableitung ist linear:}
\begin{itemize}
  \item[(i)] \( (a f)'(x_0) = a f'(x_0) \) \quad (Skalarmultiplikation)
  \item[(ii)] \( (f + g)'(x_0) = f'(x_0) + g'(x_0) \) \quad (Additivität)
\end{itemize}
Ableiten bei \(x_0\) ist eine lineare Abbildung \(T(f) := f'(x_0)\)

\textbf{Mittelwertsatz von Lagrange:}
\begin{theorem}
  Sei $f:[a,b]\to\R$ stetig auf $[a,b]$, differenzierbar auf $(a,b)$. Dann $\exists\xi\in(a,b)$ mit
  \[
    f'(\xi) = \frac{f(b)-f(a)}{b-a}.
  \]
\end{theorem}

\textbf{Mittelwertsatz:}  
\(f'(x_0)\) entspricht der Steigung der Sekante durch \((a, f(a))\) und \((b, f(b))\)

\textbf{Korollar (Folgerung aus \(f'\)):}
\begin{itemize}
  \item[(i)] \(f' = 0 \Rightarrow f\) konstant
  \item[(ii)] \(f' \ge 0 \Rightarrow f\) monoton wachsend
  \item[(iii)] \(f' > 0 \Rightarrow f\) streng monoton wachsend
\end{itemize}


\textbf{Bernoulli–de l’Hôpital:}
\begin{theorem}
  Seien $f,g$ differenzierbar in einer Umgebung von $a$ (ausgenommen evtl.\ bei $a$ selbst), und $f(a)=g(a)=0$ oder $\pm\infty$, $g'(x)\ne0$ auf dieser Umgebung (ausser evtl.\ bei $a$). Wenn
  \[
    \lim_{x\to a} \frac{f'(x)}{g'(x)} = L \in \R \cup \{\pm\infty\},
  \]
  dann
  \[
    \lim_{x\to a} \frac{f(x)}{g(x)} = L.
  \]
\end{theorem}
\begin{example}
  \[
    \lim_{x\to0} \frac{\sin x}{x} 
    \;\stackrel{\text{L'Hôpital}}= 
    \lim_{x\to0} \frac{\cos x}{1} = 1.
  \]
\end{example}

\textbf{Arkussfunktionen:}
\begin{itemize}
  \item \(\arcsin: [-1, 1] \to \left[-\tfrac{\pi}{2}, \tfrac{\pi}{2}\right], \quad \arcsin'(y) = \frac{1}{\sqrt{1 - y^2}}\)
  \item \(\arccos: [-1, 1] \to [0, \pi], \quad \arccos'(y) = \frac{-1}{\sqrt{1 - y^2}}\)
  \item \(\arctan: \mathbb{R} \to \left(-\tfrac{\pi}{2}, \tfrac{\pi}{2}\right), \quad \arctan'(y) = \frac{1}{1 + y^2}\)
\end{itemize}

\textbf{Hyperbelfunktionen:} \quad
\(\cosh x = \tfrac{e^x + e^{-x}}{2},\;
\sinh x = \tfrac{e^x - e^{-x}}{2},\;
\tanh x = \tfrac{\sinh x}{\cosh x} = \tfrac{e^x - e^{-x}}{e^x + e^{-x}}\)


\textbf{Hyperbel- und Areafunktionen:}
\begin{itemize}
  \item \(\cosh(x) = \cos(ix),\quad \sinh(x) = \sin(ix)\)
  \item \(\cosh^2 x - \sinh^2 x = 1\)
  \item \(\cosh(x+y) = \cosh x \cosh y + \sinh x \sinh y\)
  \item \(\sinh(x+y) = \sinh x \cosh y + \cosh x \sinh y\)
  \item \(\frac{d}{dx} \cosh x = \sinh x,\quad \frac{d}{dx} \sinh x = \cosh x,\quad \frac{d}{dx} \tanh x = \frac{1}{\cosh^2 x}\)
  \item Umkehrfunktionen (differenzierbar):
  \[
  \begin{aligned}
  \text{arcosh}'(y) &= \frac{1}{\sqrt{y^2 - 1}} \quad y > 1 \\
  \text{arsinh}'(y) &= \frac{1}{\sqrt{y^2 + 1}} \quad y \in \mathbb{R} \\
  \text{artanh}'(y) &= \frac{1}{1 - y^2} \quad |y| < 1
  \end{aligned}
  \]
\end{itemize}


\textbf{Trigonometrische Ableitungen:}
\begin{itemize}
  \item \(\tan': \left(-\tfrac{\pi}{2}, \tfrac{\pi}{2}\right) \to \mathbb{R}, \quad \tan'(x) = \frac{1}{\cos^2(x)}\)
\end{itemize}

\textbf{Satz (stetige Diff'barkeit eines Limes):}  
\[
f_m \xrightarrow{\text{gl.}} f,\quad f_m' \xrightarrow{\text{gl.}} g
\quad\Rightarrow\quad f \in C^1(U, \mathbb{R}^p),\; f' = g
\]


\textbf{Satz (Umkehrsatz):}  
\(f: I \to J\) diff'bar, \(f' \ne 0\)  
\(\Rightarrow\)
\begin{itemize}
  \item[(i)] \(\text{im}(f) = J\) (offenes Intervall)
  \item[(ii)] \(f\) ist bijektiv
  \item[(iii)] \(f^{-1}: J \to I\) ist diff'bar mit  
    \[
    (f^{-1})'(y) = \frac{1}{f'(f^{-1}(y))}, \quad \forall y \in J
    \]
\end{itemize}

\textbf{Taylor, 1-D:}
\begin{theorem}
  Sei $f$ $(n+1)$-mal differenzierbar im Intervall um $x_0$. Dann gilt für $x$ in einer Umgebung:
  \[
    f(x) = \sum_{k=0}^n \frac{f^{(k)}(x_0)}{k!}(x - x_0)^k 
    \;+\; R_{n+1}(x),
  \]
  wobei das Restglied in Lagrange-Form
  \[
    R_{n+1}(x) = \frac{f^{(n+1)}(\xi)}{(n+1)!}\,(x - x_0)^{n+1}
    \quad\text{für ein }\xi\text{ zwischen }x_0\text{ und }x.
  \]
\end{theorem}
\begin{example}
  $f(x) = e^x$, Entwicklung um $0$ bis Ordnung $2$:
  \[
    e^x = 1 + x + \frac{x^2}{2} + \frac{e^\xi}{6} x^3 \quad(\xi\in(0,x)).
  \]
\end{example}

\textbf{Kurvendiskussion Extremstellen (sehr sehr simpel dargestellt):}
\begin{itemize}
  \item[1.] Ableitungen: \( f',\; f'',\; f''' \)
  \item[2.] Symmetrie:
    \begin{itemize}
      \item \( f(-x) = -f(x) \) ⇒ punktsym. zum Ursprung
      \item \( f(-x) = f(x) \) ⇒ achsensym. zur y-Achse
    \end{itemize}
  \item[3.] Verhalten \( x \to \pm\infty \):  
    Nur führender Term von \( f(x) = a_n x^n + \ldots \) entscheidend:  
    \( \lim_{x \to \pm\infty} f(x) = \lim_{x \to \pm\infty} a_n x^n \)
  \item[4.] Nullstellen: \( f(x) = 0 \)
  \item[5.] Extrema/Sattelpunkte: Löse \( f'(x) = 0 \). Prüfe:
    \[
    \begin{aligned}
      f''(x_0) < 0 &\Rightarrow \text{ Maximum} \\
      f''(x_0) > 0 &\Rightarrow \text{ Minimum} \\
      f''(x_0) = 0,\; f'''(x_0) \neq 0 &\Rightarrow \text{ Sattelpunkt}
    \end{aligned}
    \]
    Falls unklar: wechsle das Vorzeichen von \( f' \) um \( x_0 \)
  \item[6.] Wendepunkte: Löse \( f''(x) = 0 \). Prüfe:
    \[
    f'''(x_0) \neq 0 \Rightarrow \text{ Wendepunkt}
    \quad \text{oder} \quad
    f'' \text{ wechselt Vorzeichen bei } x_0
    \]
  \item[7.] Graph skizzieren, Verhalten bei \( x \to \pm\infty \) beachten
\end{itemize}


% Integration___________________________________________________________________________________________________
%%%%%%
\subsection{6. Integration}

\textbf{Riemann-Integral:}  
Falls \( \lim_{n \to \infty} \Delta_n = 0 \) und  
\[
\lim_{n \to \infty} R_n = \lim_{n \to \infty} \sum_{k=1}^{n} \Delta_k f(\xi_k) =: \int_a^b f(x)\,dx
\]
existiert, dann ist \( f \) \textbf{Riemann-integrierbar} auf \( [a,b] \).

\textbf{Beispiel:} \( \displaystyle \int_0^1 (x^3 - 2x)\,dx \) mit Riemannsummen:  
\( x_k = \frac{k}{n},\; \Delta x = \frac{1}{n},\; f(x_k) = \left( \frac{k}{n} \right)^3 - 2 \cdot \frac{k}{n} \)

\[
\int_0^1 f(x)\,dx \approx \sum_{k=1}^n \frac{1}{n} f\left(\frac{k}{n}\right)
= \frac{1}{n} \sum_{k=1}^n \left( \frac{k^3}{n^3} - \frac{2k}{n} \right)
= \frac{1}{n^4} \sum k^3 - \frac{2}{n^2} \sum k
\]

\[
= \frac{1}{n^4} \left( \frac{n(n+1)}{2} \right)^2 - \frac{2}{n^2} \cdot \frac{n(n+1)}{2}
\to \frac{1}{4} - 1 = -\frac{3}{4}
\]

\textit{Check:}  
\( \int_0^1 (x^3 - 2x)\,dx = \left[ \frac{x^4}{4} - x^2 \right]_0^1 = -\frac{3}{4} \)

\textbf{Riemann-Integral (eigentlich):}
\begin{itemize}
  \item \textbf{Oberes Integral:} 
    \( \overline{\int_I} f := \inf \left\{ \int_I \psi \mid \psi \ge f,\; \psi \text{ Treppenfkt.} \right\} \)
  \item \textbf{Unteres Integral:} 
    \( \underline{\int_I} f := \sup \left\{ \int_I \varphi \mid \varphi \le f,\; \varphi \text{ Treppenfkt.} \right\} \)
  \item \textbf{f ist Riemann-integrierbar} \( \Leftrightarrow \overline{\int_I} f = \underline{\int_I} f \)
  \item \textbf{Dann:} \( \int_a^b f(x)\,dx := \int_I f := \overline{\int_I} f = \underline{\int_I} f \)
\end{itemize}

\textbf{Eigenschaften des Riemann-Integrals:}
\begin{itemize}
  \item Eingeschränkt: \( \int_a^b f := \int_I f := \int_I f|_I \), falls \( f|_I \) Riemann-integrierbar ist.

  \item Treppenfkt. \( \varphi \) ist integrierbar: \( \int_I \varphi = S_I \varphi \)

  \item Jede stetige, beschränkte \( f \) ist integrierbar.

  \item Jede beschränkte monotone \( f \) ist integrierbar.

  \item Monotonie: \( f \le g \Rightarrow \int_I f \le \int_I g \)

  \item Linearität:
  \[
    \int_I (cf) = c \int_I f, \qquad \int_I (f+g) = \int_I f + \int_I g
  \]

  \item Betrag: \( \left| \int_I f \right| \le \int_I |f| \)

  \item Additivität:
  \[
    \int_a^c f = \int_a^b f + \int_b^c f \quad \text{falls } a \le b \le c
  \]
\end{itemize}

\textbf{Riemann-Integrale mit vertauschten Grenzen:}
\[
\int_b^a f := - \int_a^b f
\]

\textbf{Gebietsadditivität (auch bei vertauschten Grenzen):}
\[
\int_a^c f = \int_a^b f + \int_b^c f
\quad \text{auch für } a, b, c \in \mathbb{R}, \text{ in der richtigen Reihenfolge.}
\]

\textbf{Rechts-/Linksdifferenzierbarkeit:}
\begin{itemize}
  \item \textbf{Rechtsseitige Ableitung:}
  \[
  f'_{+}(x_0) := \lim_{x \to x_0^+} \frac{f(x) - f(x_0)}{x - x_0}
  \]
  \item \textbf{Linksseitige Ableitung:}
  \[
  f'_{-}(x_0) := \lim_{x \to x_0^-} \frac{f(x) - f(x_0)}{x - x_0}
  \]
\end{itemize}

\textbf{Wallis-Produkt:}
\[
\frac{\pi}{2} = \lim_{n \to \infty} c_n = \frac{2 \cdot 2}{1 \cdot 3} \cdot \frac{4 \cdot 4}{3 \cdot 5} \cdot \ldots
\]


\textbf{Partielle Integration:}
\[
\int u'v = uv - \int uv'
\quad \text{(falls } u,v \text{ differenzierbar und } u'v \text{ integrierbar).}
\]

\textbf{Substitutionsregel:} Seien $F \in C^1(I),\ g \in C(F(I))$. Dann gilt:
\[
\int_{x_0}^{x_1} (g \circ F)(x) F'(x)\, dx = \int_{F(x_0)}^{F(x_1)} g(y)\, dy
\]

\textbf{Beispiel:} Berechne $\int_2^3 e^{x^2} 2x \, dx$.

Setze $g = \exp$, $F(x) = x^2$, also $F'(x) = 2x$:

\[
\int_2^3 e^{x^2} 2x\, dx = \int_{4}^{9} e^y\, dy = e^9 - e^4
\]

\textbf{Verkehrte Substitution – Bsp:}
\[
\int_{-1}^{1} \sqrt{1 - y^2} \, dy
\]
Substitution: \( y = \sin x \Rightarrow dy = \cos x \, dx \), \( x \in \left[-\tfrac{\pi}{2}, \tfrac{\pi}{2}\right] \)

\[
= \int_{-\frac{\pi}{2}}^{\frac{\pi}{2}} \sqrt{1 - \sin^2 x} \cos x \, dx = \int_{-\frac{\pi}{2}}^{\frac{\pi}{2}} \cos^2 x \, dx
\]

\[
= \frac{1}{2} \int_{-\frac{\pi}{2}}^{\frac{\pi}{2}} (1 + \cos(2x)) \, dx = \frac{1}{2} \left[ x + \frac{1}{2} \sin(2x) \right]_{-\frac{\pi}{2}}^{\frac{\pi}{2}} = \frac{1}{2} \cdot \pi = \frac{\pi}{2}
\]

\textbf{PBZ + Integral:}  
\[
\frac{1}{x^2 - 1} = \frac{1}{2} \left( \frac{1}{x - 1} - \frac{1}{x + 1} \right), \quad
\int \frac{1}{x^2 - 1} \, dx = \frac{1}{2} \log\left| \frac{x - 1}{x + 1} \right| + C
\]

\textbf{PBZ:} \quad
\frac{1}{x^2 - 1} = \frac{a}{x + 1} + \frac{b}{x - 1} 
\Rightarrow 1 = a(x - 1) + b(x + 1) 
\Rightarrow a = -\tfrac{1}{2},\; b = \tfrac{1}{2}

\textbf{Partialbruchzerlegung – Übersicht}

\begin{itemize}
  \item \textbf{Lineare Faktoren:}
    \begin{itemize}
      \item \( \frac{1}{x - a} \Rightarrow \frac{A}{x - a} \)
      \item \( \frac{1}{(x - a)^n} \Rightarrow \frac{A_1}{x - a} + \frac{A_2}{(x - a)^2} + \cdots + \frac{A_n}{(x - a)^n} \)
    \end{itemize}
    
  \item \textbf{Quadratische, irreduzible Faktoren:}
    \begin{itemize}
      \item \( \frac{1}{x^2 + bx + c} \Rightarrow \frac{Ax + B}{x^2 + bx + c} \)
      \item \( \frac{1}{(x^2 + bx + c)^n} \Rightarrow \frac{A_1x + B_1}{x^2 + bx + c} + \frac{A_2x + B_2}{(x^2 + bx + c)^2} + \cdots + \frac{A_nx + B_n}{(x^2 + bx + c)^n} \)
    \end{itemize}
\end{itemize}

\textbf{Strategien fürs Integrieren:}
\begin{itemize}
  \item \textbf{1. Vereinfachen:} Integranden umformen, z.B. kürzen oder trig. Identitäten nutzen.
  
  \item \textbf{2. Stammfunktion raten:} z.B. $f := \cos(e^x)$ \quad oder \quad $\int \frac{F'}{F} = \log|F| + C$
  
  \item \textbf{3. Produktregel rückwärts:} partielle Integration, ggf. mit Reduktionsformel und $c \int f$ auflösen.
  
  \item \textbf{4. Rationale Funktion:} → \textbf{PBZ}
  
  \item \textbf{5. Rationale Ausdrücke in $e^x$:} Substitution $F(x) := e^x$
  
  \item \textbf{6. Rationale Ausdrücke in $\cos x$, $\sin x$:} Substitution 
  \[
    F(x) := \tan\left(\frac{x}{2}\right), \quad F(x) := \tan x, \quad F(x) := \cos x \text{ od. } \sin x
  \]
  
  \item \textbf{7. Integrand mit $\sqrt{a^2 - y^2}$:} Substitution $y = F(x) := a \sin x$
\end{itemize}


