\section{Analysis I - draft}

% Grundlagen und Logik___________________________________________________________________________________________________
%%%%%%
\subsection{1. Grundlagen und Logik}

% Zahlen und Vektoren___________________________________________________________________________________________________
%%%%%%
\subsection{2. Zahlen und Vektoren}


% Folgen und Reihen ___________________________________________________________________________________________________
%%%%%%
\subsection{3. Folgen und Reihen}  
Eine \emph{ Zahlenfolge} ist eine Funktion  
\[
a : \mathbb{N}_0 \to \mathbb{C}, \quad n \mapsto a_n := a(n).
\]  

\textbf{Konvergenz einer Folge:}  
Eine Folge \( (x_n) \subset \mathbb{R} \) \textbf{konvergiert} gegen \( A \in \mathbb{R} \), falls  
\[
\forall \varepsilon > 0\; \exists N \in \mathbb{N}:\; \forall n \geq N:\; |x_n - A| < \varepsilon.
\]  
Dann schreiben wir \( \lim_{n \to \infty} x_n = A \) und nennen \( A \) den \textbf{Grenzwert}.

\[
Konvergenz \to Beschränktheit
\]  

\textbf{Beispiel Konvergent:} \( x_n = \frac{1}{n} \Rightarrow \lim\limits_{n \to \infty} x_n = 0 \)

\textbf{Beispiel Divergent:} \( x_n = n \Rightarrow \lim\limits_{n \to \infty} x_n = \infty \)

\textbf{Rechenregeln für konvergente Folgen:}  
Seien \( \lim\limits_{n \to \infty} x_n = A \), \( \lim\limits_{n \to \infty} y_n = B \), \( \alpha \in \mathbb{R} \):

\begin{itemize}
  \item \( \lim\limits_{n \to \infty} (x_n + y_n) = A + B \)
  \item \( \lim\limits_{n \to \infty} (x_n y_n) = AB \)
  \item \( \lim\limits_{n \to \infty} (\alpha x_n) = \alpha A \)
  \item Falls \( x_n \neq 0\ \forall n \), \( A \neq 0 \):  
        \( \lim\limits_{n \to \infty} \left( \frac{1}{x_n} \right) = \frac{1}{A} \)
  \item Falls \( x_n \leq y_n\ \forall n \): \( A \leq B \)
\end{itemize}

\textbf{Monotone Konvergenz:}  
Eine monotone Folge \( (x_n) \subset \mathbb{R} \) konvergiert genau dann, wenn sie beschränkt ist.

\begin{itemize}
  \item Monoton wachsend:  
  \( \lim\limits_{n \to \infty} x_n = \sup \{ x_n \mid n \in \mathbb{N} \} \)

  \item Monoton fallend:  
  \( \lim\limits_{n \to \infty} x_n = \inf \{ x_n \mid n \in \mathbb{N} \} \)
\end{itemize}

\textbf{Euler'sche Zahl:}  
\[
e := \lim\limits_{n \to \infty} \left( 1 + \frac{1}{n} \right)^n
\]

\textbf{Eigenschaften:}
\begin{itemize}
  \item \( e \) ist wohldefiniert (Folge konvergiert).
  \item \( e \approx 2{,}718\ldots \) (irrational, nicht periodisch).
\end{itemize}

\textbf{Obere/Untere Grenzwerte:}  
Für eine beschränkte Folge \( (x_n) \subset \mathbb{R} \) gilt:

\[
\limsup\limits_{n \to \infty} x_n = \lim\limits_{n \to \infty} \left( \sup_{k \geq n} x_k \right)
\]
\[
\liminf\limits_{n \to \infty} x_n = \lim\limits_{n \to \infty} \left( \inf_{k \geq n} x_k \right)
\]

\textbf{Immer:}  
\[
\liminf x_n \leq \limsup x_n
\]
\textbf{Bernoulli-de l'Hôspital:} (Abl. existiert, f und g diffbar um a und $g'(x) \not = 0$)
\[
\lim\limits_{x \to a} \frac{f(x)}{g(x)} = \lim\limits_{x \to a} \frac{f'(x)}{g'(x)}
\]
\textbf{\( e^{\log} \)–Trick:}  
\[
\lim f(x)^{g(x)} = \exp\left( \lim g(x) \cdot \log f(x) \right)
\quad \text{falls } f(x) > 0
\]

% Stetigkeit und Topologie___________________________________________________________________________________________________
%%%%%%
\subsection{4. Stetigkeit, Topologie}


% Differentialrechnung auf R___________________________________________________________________________________________________
%%%%%%
\subsection{5. Differentialrechnung auf R}

% Integration___________________________________________________________________________________________________
%%%%%%
\subsection{6. Differentialrechnung auf R}