
Sei $U \subset \mathbb{R}^n$ ein $C^1$-Gebiet und $g \in C^1(U, \mathbb{R})$ mit $\partial U = \{x \in U : g(x) = 0\}$, $\nabla g(x_0) \neq 0$.  
Dann ist die nach außen zeigende \textit{koorientierte} Einheitsnormalenabbildung
\[
\nu: \partial U \to \mathbb{R}^n, \quad \nu(x) := \frac{\nabla g(x)}{\|\nabla g(x)\|}.
\]

\textbf{Eigenschaft:}  
Die Koorientierung definiert eine Orientierung für $\partial U$ konsistent mit $U$.

\textbf{Satz von Gauß (Divergenzsatz)}

Sei $U \subseteq \mathbb{R}^n$ ein beschränktes $C^1$-Gebiet und $X \in C^1(\overline{U}, \mathbb{R}^n)$. Dann gilt:

\[
\int_U \operatorname{div} X \, dx = \int_{\partial U, \nu} X \cdot dA = \int_{\partial U} X \cdot \nu \, dA
\]

Dabei ist $\nu$ das nach aussen weisende Einheitsnormalenfeld auf $\partial U$.

\textbf{Beispiel ($n=1$)}
Für $f = X^1$ und $U = (a,b)$:

\[
\int_a^b f'(x)\, dx = f(b) - f(a)
\]

\textbf{Begründung:} Mit $X = f$ ist $\operatorname{div} X = f'$ und
\[
\int_U \operatorname{div} X \, dx = \int_{\partial U} f \cdot \nu = f(b)\cdot 1 + f(a)\cdot (-1) = f(b) - f(a)
\]

\textbf{Folgerung aus Gauß:}

Aus dem Gaußschen Integralsatz folgt:
\scalebox{0.8}{$
\int_U \operatorname{rot} X\,dx = \int_U \nabla \cdot Y\,dx = 
\int_{\partial U} Y \cdot \nu\,dA = 
\int_{\partial U} (X^2 T^2 - X^1 (-T^1))\,ds = 
\int_{\partial U} X \cdot T\,ds
$}


% Weiteres___________________________________________________________________________________________________
%%%%%%
\subsection{12. Zusatz (Integraltabellen, Bilder etc.)}


\textbf{Parametrisierungen}

\vspace{0.5em}

\textbf{Kreis:} $\Phi : (0, \infty) \times [0, 2\pi) \to \mathbb{R}^2$
\[
\Phi(r, \varphi) = \begin{pmatrix} r\cos\varphi \\ r\sin\varphi \end{pmatrix}, \quad
\det(d\Phi) = r, \quad
d\mu = r\,dr\,d\varphi
\]

\textbf{Ellipse:} $\Phi : (0, \infty) \times [0, 2\pi) \to \mathbb{R}^2$
\[
\Phi(r, \varphi) = \begin{pmatrix} ra\cos\varphi \\ rb\sin\varphi \end{pmatrix}, \quad
\det(d\Phi) = abr, \quad
d\mu = abr\,dr\,d\varphi
\]

\textbf{Zylinder:} $\Phi : (0, \infty) \times [0, 2\pi) \times \mathbb{R} \to \mathbb{R}^3$
\[
\Phi(r, \varphi, h) = \begin{pmatrix} r\cos\varphi \\ r\sin\varphi \\ h \end{pmatrix}, \quad
\det(d\Phi) = r, \quad
d\mu = r\,dr\,d\varphi\,dh
\]

\textbf{Kugel:} $\Phi : (0, \infty) \times [0, 2\pi) \times [0, \pi] \to \mathbb{R}^3$
\[
\Phi(r, \theta, \varphi) =
\begin{pmatrix}
r\cos\varphi\sin\theta \\
r\sin\varphi\sin\theta \\
r\cos\theta
\end{pmatrix}, \quad
\det(d\Phi) = r^2\sin\theta, \quad
d\mu = r^2\sin\theta\,dr\,d\varphi\,d\theta
\]

\textbf{Ellipsoid:} $\Phi : (0, \infty) \times [0, 2\pi) \times [0, \pi] \to \mathbb{R}^3$
\[
\Phi(r, \theta, \varphi) =
\begin{pmatrix}
ar\cos\varphi\sin\theta \\
br\sin\varphi\sin\theta \\
cr\cos\theta
\end{pmatrix}, \quad
\det(d\Phi) = abcr^2\sin\theta, \quad
d\mu = abcr^2\sin\theta\,dr\,d\varphi\,d\theta
\]

\vspace{1em}
\noindent
{\textbullet\ Für den Rand des Körpers muss jeweils die Parametrisierung mit einem konstanten \( r \) genommen werden.}



\begin{center}
\small
\renewcommand{\arraystretch}{1.3}
\setlength{\tabcolsep}{6pt}
\begin{tabular}{@{}ll@{}}
\multicolumn{2}{c}{\textbf{Punktmengen}} \\
\textbf{Gerade} &
\scalebox{0.85}{$M = \left\{(x,y) \in \mathbb{R}^2 \mid y = mx + q \right\}$} \\
\textbf{Kreis} &
\scalebox{0.85}{$M = \left\{(x,y) \in \mathbb{R}^2 \mid (x - x_0)^2 + (y - y_0)^2 = r^2 \right\}$} \\
\textbf{Ellipse} &
\scalebox{0.85}{$M = \left\{(x,y) \in \mathbb{R}^2 \mid \frac{(x - x_0)^2}{a^2} + \frac{(y - y_0)^2}{b^2} = 1 \right\}$} \\
\textbf{Hyperbel} &
\scalebox{0.85}{$M = \left\{(x,y) \in \mathbb{R}^2 \mid \frac{x^2}{a^2} - \frac{y^2}{b^2} = 1 \right\}$} \\
\textbf{Kugel} &
\scalebox{0.85}{$M = \left\{(x,y,z) \in \mathbb{R}^3 \mid (x - x_0)^2 + (y - y_0)^2 + (z - z_0)^2 = r^2 \right\}$} \\
\textbf{Ellipsoid} &
\scalebox{0.85}{$M = \left\{(x,y,z) \in \mathbb{R}^3 \mid \frac{(x - x_0)^2}{a^2} + \frac{(y - y_0)^2}{b^2} + \frac{(z - z_0)^2}{c^2} = 1 \right\}$} \\
\textbf{Kegel} &
\scalebox{0.85}{$M = \left\{(x,y,z) \in \mathbb{R}^3 \mid x^2 + y^2 = \frac{r^2}{h}(h - z) \right\}$} \\
\textbf{Zylinder} &
\scalebox{0.85}{$M = \left\{(x,y,z) \in \mathbb{R}^3 \mid (x - x_0)^2 + (y - y_0)^2 = r^2 \right\}$} \\
\end{tabular}
\end{center}



\begin{center}
\renewcommand{\arraystretch}{1.3}
\begin{tabular}{|>{$}c<{$}|>{$}c<{$}|>{$}c<{$}|}
\hline
f'(x) & f(x) & \int f(x)\,dx \\
\hline
0 & c & cx \\
nx^{n-1} & x^n & \frac{x^{n+1}}{n+1} \\
-\frac{1}{x^2} & \frac{1}{x} & \ln|x| \\
\frac{n}{x^{n+1}} & \frac{1}{x^n} & \frac{-1}{(n-1)x^{n-1}} \\
\frac{1}{2\sqrt{x}} & \sqrt{x} & \frac{2}{3}x^{3/2} \\
ae^{ax} & e^{ax} & \frac{1}{a}e^{ax} \\
\frac{1}{x} & \ln|x| & x(\ln x - 1) \\
\cos(x) & \sin(x) & -\cos(x) \\
-\sin(x) & \cos(x) & \sin(x) \\
\frac{1}{\cos^2(x)} & \tan(x) & -\ln|\cos(x)| \\
\cosh(x) & \sinh(x) & \cosh(x) \\
\sinh(x) & \cosh(x) & \sinh(x) \\
\frac{1}{\cosh^2(x)} & \tanh(x) & \ln(\cosh(x)) \\
\frac{1}{1+x^2} & \arctan(x) & x\arctan(x) - \frac{\ln(x^2+1)}{2}\\
\frac{1}{\sqrt{1 - x^2}} & \arcsin(x) & x \arcsin(x) + \sqrt{1-x^2}\\
-\frac{1}{\sqrt{1 - x^2}} & \arccos(x) & x \arccos(x) - \sqrt{1-x^2}\\
\frac{1}{\sqrt{1 + x^2}} & \text{arcsinh}(x) & x \text{arcsinh}(x) - \sqrt{x^2+1} \\
\frac{1}{\sqrt{x^2 - 1}} & \text{arccosh}(x) & x \text{arccosh}(x) - \sqrt{x^2-1} \\
\frac{1}{m((x+k)^2 + m^2)} & \frac{1}{(x+k)^2 + m^2} & \frac{1}{m} \arctan\left(\frac{x + k}{m}\right) \\
\hline
\end{tabular}
\end{center}

\doublebox{
\begin{minipage}{\linewidth}
\textbf{Unentscheidbare Fälle} \\
Es gilt
\[
\frac{1}{0} = \infty \qquad
\frac{1}{\infty} = 0 \qquad
\infty + \infty = \infty \qquad
0 + \infty = \infty \qquad
0^\infty = 0 \qquad
\infty^0 = \infty
\]

Folgende Fälle sind jedoch unentscheidbar:
\[
\frac{0}{0} \quad \infty - \infty \quad
1^\infty \quad 0^0 \quad 0 \cdot \infty
\]
\end{minipage}
}

\vspace{1em}

\doublebox{
\begin{minipage}{\linewidth}
\textbf{Dominanz} \\
Es gilt für $x \to +\infty$
\[
\log(\log(x)) < \log(x) < x^q < x^p < a^x < b^x < x! < x^x
\]
wobei $0 < q < p < a < b$. \\
Für $x \to 0^+$ gilt die Reihenfolge genau umgekehrt.
\end{minipage}
}

\textbf{Ansätze für inhomogene DGLs:}

\begin{center}
\renewcommand{\arraystretch}{1.4}
\begin{tabular}{|
  >{\centering\arraybackslash}m{4cm}|
  >{\centering\arraybackslash}m{4cm}|
}
\hline
$g(t)$ & $y_{\text{part}}(t)$ \\
\hline
$C$ & $A$ \\
$Ce^{at}$ & $Ae^{at}$ \\
$C\cos(bt)$ & $A\sin(bt) + B\cos(bt)$ \\
$C\sin(bt)$ & $A\sin(bt) + B\cos(bt)$ \\
$C\cos(bt)e^{at}$ & $(A\sin(bt) + B\cos(bt))e^{at}$ \\
$C\sin(bt)e^{at}$ & $(A\sin(bt) + B\cos(bt))e^{at}$ \\
$k_nt^n + \cdots + k_0$ & $K_nt^n + \cdots + K_0$ \\
$(k_nt^n + \cdots + k_0)e^{at}$ & $(K_nt^n + \cdots + K_0)e^{at}$ \\
$(k_nt^n + \cdots + k_0)\cos(bt)$ &
$\begin{aligned}
&(K_nt^n + \cdots)\sin(bt)\\
&+ (M_nt^n + \cdots)\cos(bt)
\end{aligned}$ \\
$(k_nt^n + \cdots + k_0)\sin(bt)$ &
$\begin{aligned}
&(K_nt^n + \cdots)\sin(bt)\\
&+ (M_nt^n + \cdots)\cos(bt)
\end{aligned}$ \\
\hline
\end{tabular}
\end{center}
