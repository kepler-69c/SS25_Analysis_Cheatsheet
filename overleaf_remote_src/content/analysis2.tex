\section{Analysis II - draft}

% GDG (1. Ordnung)___________________________________________________________________________________________________
%%%%%%
\subsection{7. GDG (1. Ordnung)}

\begin{itemize}
  \item gewöhnlich: \quad \( F(x, y(x), y'(x), y''(x), \dots) = 0 \)
  \item ordnung: \quad ordnung der höchsten ableitung
  \item linear: \quad \( p_n(x)y^{(n)} + \dots + p_1(x)y' + p_0(x)y + g(x) = 0 \)
  \item homogen: \quad \( p_n(x)y^{(n)} + \dots + p_1(x)y' + p_0(x)y = 0 \)
\end{itemize}

\textbf{Superpositionsprinzip:} \quad
jede lineare kombination von lösungen einer homogenen GDG ist ebenfalls lösung der GDG.

\textbf{Satz (Lösungsraum):}

\[
\dim \mathcal{Z} = n \quad / \quad \dim \mathcal{Z}_{\mathbb{R}} = n
\]

\begin{itemize}
  \item[(i)] die menge 
    \[
    \mathcal{Z} := \left\{ f \in \mathcal{C}^n(\mathbb{R}; \mathbb{C}) \;\middle|\; f \text{ löst } \sum_{i=0}^{n} a_i(x) f^{(i)}(x) = 0 \right\}
    \]
    ist ein komplexer vektorraum.
    
  \item[(ii)] die menge 
    \[
    \mathcal{Z} := \left\{ f \in \mathcal{C}^n(\mathbb{R}; \mathbb{R}) \;\middle|\; f \text{ löst } \sum_{i=0}^{n} a_i(x) f^{(i)}(x) = 0 \right\}
    \]
    ist ein reeller vektorraum.
\end{itemize}

\textbf{GDG der Form:} \quad
$f^{(n)} + a_{n-1} f^{(n-1)} + a_{n-2} f^{(n-2)} + \dots + a_0 f = 0$

\textbf{Ansatz:} \quad $f := e^{\lambda t}$

\textbf{Satz (Basis für den Lösungsraum):} \quad
die funktionen \( f_{ik}(t) := t^k e^{\lambda_i t} \), \quad \( 1 \leq i \leq \ell, \; 0 \leq k \leq m_i \) \\
bilden eine basis des komplexen vektorraums
\[
\mathcal{Z} := \left\{ f \in \mathcal{C}^n(\mathbb{R}; \mathbb{C}) \;\middle|\; f \text{ löst } \sum_{i=0}^{n} a_i f^{(i)} = 0 \right\}
\]

\textbf{Beispiel:} \quad
\[
\ddot{f} - 2\dot{f} + f = 0 \quad \Rightarrow \quad f(t) = c_1 e^t + c_2 t e^t
\]

\vspace{1em}

\textcolor{orange}{\textbf{Zweite Ordnung}} \quad
\textbf{GDG der Form:} \quad $\ddot{f} + a\_1 \dot{f} + a\_0 f = 0$ \qquad \text{(schwingkreis)}

\[
\omega_0 := \sqrt{a_0}, \qquad \delta := \frac{a_1}{2}
\quad \Rightarrow \quad
\ddot{f} + 2\delta \dot{f} + \omega_0^2 f = 0
\]

\[
\Rightarrow \quad \mu := \sqrt{\delta^2 - \omega_0^2}, \qquad \omega_d := \sqrt{\omega_0^2 - \delta^2}
\]

\begin{itemize}
  \item[1)] \( \delta < \omega_0 \): \quad
    \( f(t) = e^{-\delta t} \left( a \cos(\omega_d t) + b \sin(\omega_d t) \right) \)

  \item[2)] \( \delta = \omega_0 \): \quad
    \( f(t) = (c_1 + c_2 t) e^{-\delta t} \)

  \item[3)] \( \delta > \omega_0 \): \quad
    \( f(t) = c_1 e^{(-\delta + \mu)t} + c_2 e^{(-\delta - \mu)t} \)
\end{itemize}


% Differentialrechnung im R___________________________________________________________________________________________________
%%%%%%
\subsection{8. Differentialrechnung im R}

% Umkehrsatz, implizite Funktionen___________________________________________________________________________________________________
%%%%%%
\subsection{9. Umkehrsatz, implizite Funktionen, Untermannigfaltigkeit, Tangentialraum}

% Mehrdimensionale Riemann-integration,Satz von Fubini über wiederholte Integration, Jordan-Mass, Substitutionsregel für mehrdimensionale Integrale___________________________________________________________________________________________________
%%%%%%
\subsection{10. Mehrdimensionale Riemann integration, Satz von Fubini über wiederholte Integration, Jordan-Mass, Substitutionsregel für mehrdimensionale Integrale}


% Vektorfelder und die Sätze von Green, Stokes und Gauss (WICHTIG!!!)___________________________________________________________________________________________________
%%%%%%
\subsection{11. Vektorfelder und die 
Sätze von Green, Stokes und Gauss}

% Weiteres___________________________________________________________________________________________________
%%%%%%
\subsection{12. Zusatz (Integraltabellen, Bilder etc.}